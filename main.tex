\documentclass{article}
\usepackage[utf8]{inputenc}
\usepackage{array}
\usepackage{graphicx}
\usepackage{multirow}
\usepackage{subcaption}
\usepackage{color, colortbl}
\usepackage{float}
\usepackage{hyperref}
\usepackage{setspace}
\usepackage{lipsum}

\usepackage{fancyhdr}
\pagestyle{fancy}

\newcommand{\SubItem}[1]{
    {\setlength\itemindent{15pt} \item[-] #1}
}

\newcolumntype{P}[1]{>{\centering\arraybackslash}p{#1}}
\newcolumntype{M}[1]{>{\centering\arraybackslash}m{#1}}

\title{wrs_rulebook}
\date{May 2019}

\makeatletter
\newcommand\abtname{About this Rulebook}
\if@titlepage
   \newenvironment{about}{%
       \titlepage
       \null\vfil
       \@beginparpenalty\@lowpenalty
       \begin{center}%
         \bfseries \abtname
         \@endparpenalty\@M
       \end{center}}%
      {\par\vfil\null\endtitlepage}
\else
   \newenvironment{about}{%
       \if@twocolumn
         \section*{\abstractname}%
       \else
         \small
         \begin{center}%
           {\bfseries \abtname\vspace{-.5em}\vspace{\z@}}%
         \end{center}%
         \quotation
       \fi}
       {\if@twocolumn\else\endquotation\fi}
\fi

\newcommand\ackname{Acknowledgements}
\if@titlepage
   \newenvironment{acknowledgements}{%
       \titlepage
       \null\vfil
       \@beginparpenalty\@lowpenalty
       \begin{center}%
         \bfseries \ackname
         \@endparpenalty\@M
       \end{center}}%
      {\par\vfil\null\endtitlepage}
\else
   \newenvironment{acknowledgements}{%
       \if@twocolumn
         \section*{\abstractname}%
       \else
         \small
         \begin{center}%
           {\bfseries \ackname\vspace{-.5em}\vspace{\z@}}%
         \end{center}%
         \quotation
       \fi}
       {\if@twocolumn\else\endquotation\fi}
\fi
\makeatother

\rhead{\includegraphics[width=2cm]{{images/header.png}}}

\begin{document}

\begin{titlepage}
	\centering
	%\includegraphics[width=0.15\textwidth]{example-image-1x1}\par\vspace{1cm}
	{\scshape\LARGE World Robot Summit 2020 \par}
	\vspace{1cm}
	{\scshape\Large Partner Robot Challenge\par}
	\vspace{1.5cm}
	{\huge\bfseries Real Space\par}
	\vspace{2cm}
	{\Large\itshape Rules \& Regulations\par}
	\vfill
	\includegraphics[width=0.50\textwidth]{{images/real01.png}}
	\vfill
	
% Bottom of the page
%	{\large \today\par}
{\large Version 1.0.0 (January 1, 2020)}
\end{titlepage}

\newpage

\begin{about}

This is the official rulebook of the Partner Robot Challenge (Real Space) 2020 competition. It has been written by the Service Category Technical Committee members.

\end{about}

\begin{acknowledgements}

We would like to thank all the people who contributed to the World Robot Summit Service Category Technical Committee members with their feedback and comments. We also like to thank the members of RoboCup@Home community.

testtesttest
\end{acknowledgements}

\newpage
%\doublespacing
\spacing{1.5}
\tableofcontents{}
\singlespacing
\newpage

\section{Introduction}

\subsection{Service Robotics Category}

The Competition will be helpful to solve social problems stemming from rapidly aging population and declining birth rates through technology that works alongside humans to provide a variety of services. As we enter the age in which robots will become a part of people's lives, there is a need for robots that can perform a variety of services in cooperation with humans. There is a need for service robots that can work safely and reliably with people and for technologies that create the environment necessary for developing such robots. These include AI learning through which humans and robots engage in advanced communication, Big Data information-sharing through clouds, collection and use of information gathered through IoT technology, etc. Human resource development (training) is indispensable in the development of robotic technology and social implementation of robots.

\subsection{Partner Robot Challenge}

The concept of Partner Robot Challenge is to foster the collaboration between human and robot. Not limited to assistance for handicapped person, but also in domestic home environment with living children and elderly person, it is targeted to realize a rich collaborative living environment for human and robot. It is not the aim of this competition for the robot to complete the task alone, but with the communication and workload sharing with human user, this challenge competes the technologies for smooth collaboration between human and robot. The focus on human-robot collaboration is the uniqueness of this competition.

\newpage
\section{Partner Robot Challenge \\Real Space}

\subsection{Concept}

The concept behind the competition is adapted from the roles of a home assistant. A home assistant is trained to aid or assist human on daily house chores. Apart from physical work assistance, the home assistant can also maintain friendship and social companionship with human. The fundamental concept of this challenge is to develop a robot that resembles a home assistant as a partner to support human.

In the competition, the robot is to compete for the task as a home assistant. Not limited to the support for handicapped person, the robot is anticipated to be able to perform a wide range of tasks as a partner to elderly person and normal healthy person too.

\begin{figure}[!h]%[!htbp]
	\centering
	%\begin{center}
	\includegraphics[width=0.80\textwidth]{{images/real02.png}}
	\caption{Concept of the Partner Robot Challenge.}
	\label{fig:concept}
	%\end{center}
\end{figure}

\subsection{Competition}

The competition consists of a qualifier round, semi-final, and a final. 

The objective is to evaluate the fundamental skills of the robot to accomplish basic functions of a home assistant, such as serving items, searching items, and tidy-up the environment. The robot (with human collaboration) is required to perform such tasks within a set of defined rules and environments in order to acquire scores for the competition. The competition ends with the Final where only the two highest ranked teams compete to become the winner. 

In order to stimulate research through competitions, competitions should be recognised as a useful benchmark for research outputs. Currently, the layout of the arena in the many home service robot competitions, such as RoboCup, RockIn, and WRS, is different between them, is too large, and changes every year, and therefore it is virtually impossible to prepare the same setup in a research laboratory. In addition, the tasks and the rules change constantly and objective comparison between solutions becomes difficult. One of the aims of this competition is to provide a benchmark for home service robots where different research groups can compare their different solutions to a given problem.

%The participating teams are encouraged to demonstrate their latest research development on new approaches and applications of the assistance robot in an interesting scenario setting. 
A primary objective of the competition is to observe all the different solutions from the many teams to the very same problem.

\subsection{Standard Robot Platform}
The Human Support Robot\footnotemark (HSR) by Toyota Motor Corporation will be used as the Standard Platform Robot for the Partner Robot Challenge (Real Space).

\footnotetext[1]{Takashi Yamamoto et. al. \textit{Development of Human Support Robot as the research platform of a domestic mobile manipulator}. ROBOMECH Journal, 2019.}





\newpage
\section{General Rules \\Tidy Up Here}

\textbf{Description:} Move the objects spread around a room to their designated locations and provide a person with something to eat when requested.

\textbf{Time:} 15 min for task 1 + 5 min for task 2

\subsection{Objectives}
\begin{itemize}
    \item Task 1: Tidy up objects from the incorrect positions to a predetermined tidy up space.
    \item Task 2: Provide a person within a group with some food from a shelf when requested while avoiding obstacles when navigating.
\end{itemize}

\subsection{Technical Focus}
\begin{itemize}
    \item Semantic mapping, unknown object perception and manipulation, path planning.
    \item HRI, semantic mapping, object manipulation, motion and path planning.
\end{itemize}

\subsection{Vision}

By providing an easy to understand challenge, we aim at identifying and solving front-end problems that might arise when using a service robot in indoors environments, such as safety and stability, autonomous error recovery, task performance and time consumption, natural motion and path planing, etc.

The Key Performance Indicator is based on a 4S philosophy: Speed, Smooth/ Smart, Stable, and Safe.

\subsection{Settings}

\textbf{Environment}
\begin{itemize}
    \item The room layout and furniture positions will be as in as in Figure \ref{fig:arena} and the furniture models are presented in Appendix \ref{sec:furniture}.
    \item Each Task is carried out in the following rooms:
        \SubItem{Room 1: Living room (Task 1), a 3.5x4.0 meters area, with a 1.2m long door in the top left side of the wall working as a main entrance and a 1.2m long free access (i.e. without door) in the bottom of the shared wall connecting to the dining room, as shown in Figure \ref{fig:tidyup}.}
        \SubItem{Room 2: Dining room (Task 2), a 2.5x4.0 meters area, as illustrated in Figure \ref{fig:bringdrink}.}
        
    \item The operator sends instructions from the front entrance room, including what object to deliver in Task 2.
    \item Furniture:
        \SubItem{Task 1: stairs-like drawer (Figure \ref{fig:drawer}), long table (x2), tall table, bin (x2).}
        \SubItem{Task 2: shelf without door (Figure \ref{fig:shelf}), chair (x2).}
    \item Storage places:
        \SubItem{Task 1: drawer, tray, bin, pen pencil holder (Figure \ref{fig:containers}).}
        \SubItem{Task 2: shelf without door.}
\end{itemize}

\begin{figure}[!h]%[!htbp]
	\centering
	%\begin{center}
	\includegraphics[width=0.80\textwidth]{{images/map.png}}
	\caption{Test arena layout. The red arrow indicates the starting point at the beginning of the competition or after a restart.}
	\label{fig:arena}
	%\end{center}
\end{figure}

\begin{figure}[!h]
	\centering
	\begin{subfigure}{.45\textwidth}
  		\centering
  		\includegraphics[width=1\textwidth]{{images/tidy_up.jpg}}
  		\caption{}
  		\label{fig:tidyup}
	\end{subfigure}
	\begin{subfigure}{.45\textwidth}
  		\centering
  		\includegraphics[width=1\textwidth]{{images/bring_drink.jpg}}
  		\caption{}
  		\label{fig:bringdrink}
	\end{subfigure}
	\caption{a) Living room where objects are scattered about. b) Dining room with a shelf and two chairs. For ease of visibility for the audience, \textbf{the wall will have a minimum height of 60 cm}.}
	\label{fig:wrschallenge}
\end{figure}

\begin{figure}[!h]%[!htbp]
	\centering
	\begin{subfigure}{.45\textwidth}
  		\centering
	    \includegraphics[width=1\textwidth]{{images/drawer.jpg}}
	    \caption{}
  		\label{fig:drawer}
	\end{subfigure}
	\begin{subfigure}{.45\textwidth}
  		\centering
  		\includegraphics[width=1\textwidth]{{images/handles_b.jpg}}
  		\caption{}
  		\label{fig:handles}
	\end{subfigure}
	\caption{a) Stairs-like drawer. The drawer will be as it is (left drawer) or it might have a handle and/or be opened under team request (middle drawers); when opening a drawer, it should not be unmounted from the shelf. b) Several views of the easy-to-manipulate handles (\url{https://www.amazon.com/Suction-BYBYCD-Handles-Bathroom-Adsorbent/dp/B07GYKJCTR}).}%; as a reference, the circular base has a diameter of 6cm and a height of 2cm while the handle part has a long side of 6cm, short side of 3.5cm and a height of 3cm.}
	\label{fig:fulldrawer}
	%\end{center}
\end{figure}

\begin{figure}[!h]
	\centering
	%\begin{subfigure}{.45\textwidth}
  		%\centering
  		\includegraphics[width=0.45\textwidth]{{images/shelf.jpg}}
  		%\caption{}
  		%\label{fig:objects}
	%\end{subfigure}
	%\begin{subfigure}{.45\textwidth}
  	%	\centering
  	%	\includegraphics[width=1\textwidth]{{images/lateral_door_b.jpg}}
  	%	\caption{}
  	%	\label{fig:door}
	%\end{subfigure}
	\caption{Front view of the shelf and objects inside it.}
	\label{fig:shelf}
\end{figure}

\begin{figure}[!h]%[!htbp]
	\centering
	%\begin{center}
	\includegraphics[width=0.80\textwidth]{{images/tray_container.jpg}}
	\caption{Containers may vary in shape and size, but they will be within a predefined area in the deposit area.}
	\label{fig:containers}
	%\end{center}
\end{figure}

\textbf{Objects}
\begin{itemize}
    \item Known objects - with pre-announced object recognition data and location information --, 25 units in 5 categories\footnotemark (each category has 5 units randomly chosen from the dataset; the selection process is performed per test): 
        \SubItem{Food (e.g. chips can, coffee can, cracker box), kitchen items (e.g. glass, bowl, mug), tools (e.g. clamps, padlock's keys), shape items (e.g. baseball, tennis ball), and task items (e.g. Rubick’s cube, t-shirt, airplane toy)}
    \item Orientation-based items. It is \textbf{important} to notices the existence of this property, where exclusively the following objects have it: 
        \SubItem{small marker: tip faces downwards}
        \SubItem{large marker: tip faces downwards}
        %\SubItem{scissors: tip of blade faces downwards}
        %\SubItem{knife: handle faces upwards}
        \SubItem{fork: handle faces upwards}
        \SubItem{spoon: handle faces upwards}
    \item Unknown objects - without pre-announced object recognition data nor location information. They will be frequently used daily goods and tools as in Young Sang Choi et. al. \footnotemark, and may include deformable objects such as clothing, food, magazines, etc., 5 units (one per category). 
\end{itemize}

\footnotetext[3]{Berk Calli et. al. \textit{Benchmarking in Manipulation Research: The YCB Object and Model Set and Benchmarking Protocols}. IEEE Robotics and Automation Magazine, 2015.}

\footnotetext[4]{Young Sang Choi et. al. (2009), \textit{A list of household objects for robotic retrieval prioritized by people with ALS}. IEEE International Conference on Rehabilitation Robotics, ICORR 2009.}

%\footnotetext[3]{http://www.ycbbenchmarks.com/}

Object's specifications and their corresponding 3D models can be found in http://www.ycbbenchmarks.com/ .

\textbf{Boss-character}. During the competitions, one local and representative object may be used to encourage teams to solve some of the most difficult tasks while making the competition more exciting for visitors.\\

\textbf{Tidy Up}
\begin{itemize}
    \item The name and location of the deposits are fixed, such as tools drawer, food tray, task items bin, kitchen items container, etc.
    \item Known objects are given with pre-announced information on their corresponding deposits. Orientation-based items can be placed in a deposit corresponding to their category (correct category point) or property (correct category and, if correctly placed, orientation points).
    \item For unknown objects, the robot inquires the operator to get information on tidy up space (by voice or QR code), or by the robot own judgement (e.g. toys go into task items bin or food tray?).
    \item The furniture doors will have easy-to-manipulate handles (as in Figure \ref{fig:handles}).
\end{itemize}

\subsection{Before the Competition}

\begin{itemize}
    \item Teams are given the time to create/adjust map data in the actual apartment (no objects are placed). The goal location for Task 2a will be announced.
    \item Each team will be given setup time in turns during the setup day.
    \item Right before the competition, the referee sets the rooms into messy condition (within a specified error tolerance range).
    \item Area 1. Living room (task planning, object recognition and manipulation): for Task 1
        \SubItem{30 objects (5 categories, 5 known + 1 unknown objects per category) are randomly placed in the room (on the floor and or tables).}
        \SubItem{1 boss-character placed at a challenging position within the arena (please refer to Section \ref{sec:scores} for details).}
    \item Area 2. Dining space (HRI, path and motion planning): for Task 2
        \SubItem{Obstacles on the floor are set in the access between rooms. Several food items are placed inside the shelf. Some people are in the dining room.}
\end{itemize}

\subsection{During the Competition}

The robot enters the living room in a messy condition, tidies up the room in 15 minutes, then moves to the dining room while avoiding obstacles on the floor and delivers the food item within 5 minutes.

Several teams are expected to participate and two identical arenas are set up. From all teams, two groups are generated and a group-tournament is considered, with quarter-finals, semi-finals and a final. Each team competes against all others in their bracket, where two teams perform the same task in parallel (Arena A and Arena B). A match winner gets 3 points, a draw gives 1 point to each team, and a defeat makes 0 points. After the qualifiers, the top-ranked teams progress to the next round. A detailed description of a full match is described below:

\begin{itemize}
    \item Each test starts with the operator's command in the Living Room at the front entrance (starting point); the command includes the object to be delivered in Task 2. The start signal is the door opening by the referee.
    \item When Task 1 has been completed or time limit is up, the robot moves to Room 2 to perform Task 2. As the name of the task suggests, the robot should move at least one object to a deposit before being able to move to the next room or being granted any bonus point. The robot should inform the operator whether the tasks have been completed or the time limit is up.
    \item Task 1
        \SubItem{After indicating the object to be delivered in Task 2, start the competition by opening the entrance door by the referee.}
        \SubItem{Perform tidy up.}
        \SubItem{Moves to Room 2 after completion of the task or the time is over (15 min). It is important to notice that Task 2 should be completed in 5 minutes after entering Room 2, in case the robot enters before the 15 min mark, and by the 20 minutes limit, in case the robot enters to Room 2 after the 15 min mark.}
    \item Task 2a
        \SubItem{Enters the dining room and goes to a goal location while avoiding obstacles.}
    \item Task 2b
        \SubItem{The robot goes directly to the shelf and takes the right object given at the beginning of the test.}
        \SubItem{The robot moves toward the group of persons and gives the object to the one requesting it.}
        \SubItem{The test ends with the robot reporting the completion of the task or the time is over.}
\end{itemize}

\subsection{Scores} \label{sec:scores}

\begin{itemize}
    \item Task 1
        \SubItem{Grasping an object inside the room and correctly depositing it in a location. It is considered a successful grasping if the robot is able to bring the object to a deposit area; if the robot drops the object in its way to a deposit, it is considered unsuccessful and will be subject to a penalisation (hint: the robot can take a dropped object again); a successful deposit consist on the object being inside the deposit, as in Figure \ref{fig:delivery}. Special consideration is given to clothing and textiles, where they can hang in the deposit's border, but without touching the floor (Figure \ref{fig:cloths}): 10 points x 30 units.}
        \SubItem{Correctly depositing an object according to its category: 10 additional points per object.}
        \SubItem{Correctly depositing an object based on its orientation (e.g. markers' tips facing downwards as in Figure \ref{fig:orientation}): 10 additional points per object.}
        \SubItem{Correctly finishing the task within the time limit (i.e. all objects in the search area have been placed in their corresponding deposit): 50 points.}
    \item Task 2a
        \SubItem{Navigating to a goal location in the next room while avoiding random obstacles lying on the floor within the access area. As we are considering very small and flat objects, a collision will be reckoned if there is physical contact with any object on the floor: 100 points.}
    \item Task 2b
        \SubItem{Grasping any food item in the shelf: 30 points x 1 unit.}
        \SubItem{Correctly taking the requested food item among many objects in the shelf: 70 additional points. Note that the requested object is likely to be occluded by other items in the shelf and therefore a grasping strategy should be implemented (e.g. first move item one and item two in front of the target object and without hitting any other object in the shelf before being able to take the requested object).}
        \SubItem{Delivering the object to a person in the delivery area: 30 points x 1 person.}
        \SubItem{Correctly detecting a person request and giving the food item to her/him: 70 additional points.}
        \SubItem{Correctly finishing the task within time limit (i.e. correct object delivered to the right person and without collisions): 50 points.}
        \SubItem{If any time remaining, add 20 points per minute.}
    \item The final score in a full round is the sum of partial scores per match in their corresponding bracket.
\end{itemize}

\begin{figure}[!h]
\centering
	\begin{subfigure}{.30\textwidth}
  		\centering
  		\includegraphics[width=1\textwidth]{{images/tray_setup.jpg}}
  		\caption{}
  		\label{fig:deliverysetup}
	\end{subfigure}
	\begin{subfigure}{.30\textwidth}
  		\centering
  		\includegraphics[width=1\textwidth]{{images/tray_correct.jpg}}
  		\caption{}
  		\label{fig:deliverycorrect}
	\end{subfigure}
		\begin{subfigure}{.30\textwidth}
  		\centering
  		\includegraphics[width=1\textwidth]{{images/tray_wrong.jpg}}
  		\caption{}
  		\label{fig:deliverywrong}
	\end{subfigure}
	\caption{Example of a delivery process in a tray case: a) given a target deposit with objects in it, b) it is considered a correct delivery if the objects is placed softly in a free space, and c) a hit if it is placed in a occupied space. In any case, if, in its final position, a given object touches the table or it doesn't touch the tray (a piling situation where one object is over another), it is considered a failed delivery (zero points for that object).}
	\label{fig:delivery} 
\end{figure}

\begin{figure}[!h]
	\centering
	\begin{subfigure}{.45\textwidth}
  		\centering
  		\includegraphics[width=1\textwidth]{{images/cloths_correct.jpg}}
  		\caption{}
  		\label{fig:clothscorrect}
	\end{subfigure}
	\begin{subfigure}{.45\textwidth}
  		\centering
  		\includegraphics[width=1\textwidth]{{images/cloths_wrong.jpg}}
  		\caption{}
  		\label{fig:clothswrong}
	\end{subfigure}
	\caption{In the case of drawers, containers, and bins, a correct delivery is considered if the objects are inside them in any position. In the special case of clothes, it is consider a) a correct delivery (no contact with the floor) and b) an incorrect delivery.}
	\label{fig:cloths}
\end{figure}

\begin{figure}[!h]%[!htbp]
\centering
	\begin{subfigure}{.30\textwidth}
  		\centering
  		\includegraphics[width=1\textwidth]{{images/orientation_good_1.jpg}}
  		\caption{}
  		\label{fig:orientationgood1}
	\end{subfigure}
	\begin{subfigure}{.30\textwidth}
  		\centering
  		\includegraphics[width=1\textwidth]{{images/orientation_good_2.jpg}}
  		\caption{}
  		\label{fig:orientationgood2}
	\end{subfigure}
		\begin{subfigure}{.30\textwidth}
  		\centering
  		\includegraphics[width=1\textwidth]{{images/orientation_bad_1.jpg}}
  		\caption{}
  		\label{fig:orientationbad1}
	\end{subfigure}
	\caption{Example of a correct a) markers and b) cutlery orientation-based items' delivery, and c) incorrect delivery, where the tips should face downwards.}
	\label{fig:orientation}
\end{figure}

\begin{itemize}
    \item Bonus points can be given if the team opts for the bonus challenges (50 points per challenge):
        \SubItem{Boss-character manipulation, this includes one single object in a difficult place, like at the back of or under a table.%, and may be textureless and or flat (see Figure \ref{fig:difficult}).
        The boss-object will be released in due time to the qualified teams.}
        \SubItem{Opening the three shelf drawers. No further human interaction is allowed. If the robot opens the drawers such that no object can fit in there and therefore the referee or the operator needs to open it more, no bonus points will be considered.}
\end{itemize}

\iffalse
\begin{figure}[!h]
	\centering
	\begin{subfigure}{.45\textwidth}
  		\centering
  		\includegraphics[width=1\textwidth]{{images/difficult_1.jpg}}
  		\caption{}
  		\label{fig:difficult1}
	\end{subfigure}
	\begin{subfigure}{.45\textwidth}
  		\centering
  		\includegraphics[width=1\textwidth]{{images/difficult_2.jpg}}
  		\caption{}
  		\label{fig:difficult2}
	\end{subfigure}
	\caption{Hard-to-manipulate objects include a) flat, textureless, and transparent objects, and b) objects in difficult places such as under tables.}
	\label{fig:difficult}
\end{figure}
\fi

\begin{itemize}
    \item Penalty points in \textbf{Task 1} can be given in the following scenarios:
        %\SubItem{The robot moves to room 2 before the task is completed (i.e. placing at least one object into their correct location) or the time limit ($\pm$ 15 seconds) is reached: -100 points.}
        \SubItem{Dropping an object (however, if the robot successfully re-takes the same object, no penalisation will be considered): Points x 50\% in the next object.}
        \SubItem{If the robot is stuck during the competition, it can be restarted if necessary. After the robot has been stopped by the operator, the robot should be operated and initialised in the starting point, limited to a maximum of two restarts  (in total with Task 2,to prevent repeatedly fine adjustments): Points x 50\% in the next object.}
        \SubItem{The robot (except for the manipulator's fingers and palm, see Figure \ref{fig:handhitareas}) or the additional equipment prepared by the team hits any object/obstacle inside the arena or in the shelf, or any part of the robot hits the furniture or walls: Points x 50\% for the current object.}
        \SubItem{The robot performs a false delivery (i.e. moves to the deposit location without holding any object in its manipulator): Points x0.0 for the current object.}
    \item Similarly, penalty points in \textbf{Task 2} are considered in the following scenarios:
        \SubItem{Dropping an object out of the shelf (however, if the robot successfully re-takes the same object and delivers it back inside the shelf, no penalisation will be considered): 0.25 x Partial points in Task 2b$_1$ (considered as a HIT in the score sheet).}
        \SubItem{If the robot is stuck during the competition, it can be restarted if necessary. After the robot has been stopped by the operator, the robot should be operated and initialised in the starting point, limited to a maximum of two restarts (in total with Task 1, to prevent repeatedly fine adjustments): 0.25 x Partial points in Task 2b$_1$ or Task 2b$_2$ (considered as a HIT in the score sheet).}
        \SubItem{The robot (except for the manipulator's fingers and palm, see Figure \ref{fig:handhitareas}) or the additional equipment prepared by the team hits any object/obstacle inside the arena or in the shelf, or any part of the robot hits the furniture or walls: 0.25 x Partial points in Task 2b$_1$ or Task 2b$_2$ (considered as a HIT in the score sheet).}
        \SubItem{The robot performs a false delivery (i.e. moves to the deposit location without holding any object in its manipulator): Points x0.0 in Task 2b.}
\end{itemize}

IMPORTANT: \textbf{Emergency stop.} If the referees or the TC observe behaviours out of the 4S philosophy (Speed, Smooth/Smart, Stable, Safe) such as the robot constantly hitting objects while navigating or grasping objects (i.e. no path and or motion planning is being used), repeatedly dropping objects (i.e. no intelligent manipulation strategies are in use), and so on, they can stop the test and grant zero points for that trial to the team.

\begin{figure}[!h]
\centering
	\begin{subfigure}{.30\textwidth}
  		\centering
  		\includegraphics[width=1\textwidth]{{images/hand_a.jpg}}
  		\caption{}
  		\label{fig:handareas}
	\end{subfigure}
	\begin{subfigure}{.30\textwidth}
  		\centering
  		\includegraphics[width=1\textwidth]{{images/hand_b.jpg}}
  		\caption{}
  		\label{fig:handnohit}
	\end{subfigure} \\
	\begin{subfigure}{.30\textwidth}
  		\centering
  		\includegraphics[width=1\textwidth]{{images/hand_c.jpg}}
  		\caption{}
  		\label{fig:handhit}
	\end{subfigure}
	\begin{subfigure}{.30\textwidth}
  		\centering
  		\includegraphics[width=1\textwidth]{{images/hand_d.jpg}}
  		\caption{}
  		\label{fig:handnohit}
	\end{subfigure}
	\caption{a) Valid manipulator's fingers and palm areas that can be in contact with an object without being considered a hit. b) In the example, to take the apple surrounded by two objects, c) it is allowed to move the objects around the target object without dropping them or hitting and moving neighbour objects with the displaced objects. d) A hit is considered when the robot hand (except the fingers and palm) or any other part of the robot touches an object (e.g. the pear in the image) or when the center of mass of a neighbour object is displaced two or more centimeters after a hit by a displaced object (e.g. the boxes were significantly moved after a hit by the peach).}
	\label{fig:handhitareas} 
\end{figure}

\subsection{Additional rules and remarks}

\begin{itemize}
    \item The use of an external laptop mounted in the back of the robot is allowed.
    \item The robot can use any type of external computing (e.g. cloud computing) and can connect to it via wireless (WiFi provided, but not guaranteed).% or Ethernet cable (an external PC can be placed in a designated area) [Note: use of a Ethernet cable TO BE DECIDED].
    %\item An external camera in a designated area inside the arena at a high height and facing towards the floor can be used (not provided by the organizers). All the external equipment should fit in the designated area (see Figure \ref{fig:external}).
    \item Hand manipulator can be changed (i.e. only the gripper design), TOYOTA will provide the drawings and manuals under request. Safety and motion tests should be confirmed by each team prior and during the competition.
    %\item The use of external tools (carts, trays, etc.) is allowed (no actuators or active devices) to deliver multiple objects at the same time.
    \item Passive sensors (e.g. depth sensors mounted in the hand, such as lasers or sonars) are allowed within the restrictions to be provided by TOYOTA. However, no actuators are allowed (active devices).
    \item Random robot inspections will be performed during the competition.
\end{itemize}

IMPORTANT: \textbf{External devices notification.} Before using external tools or performing any hardware modification, teams should submit to the committee a letter of intent specifying the details of such tool or modification to receive approval and or feedback.

%\begin{figure}[!h]%[!htbp]
%	\centering
%	%\begin{center}
%	\includegraphics[width=0.80\textwidth]{{images/external_device.jpg}}
%	\caption{Designated area (50x50 cm) inside the arena to place an external camera.}
%	\label{fig:external}
%	%\end{center}
%\end{figure}

%\subsection{Data Recordings}

%The following data may be required by the referee to support scoring.
%\begin{itemize}
%    \item Recognised object images.
%    \item List of manipulated objects.
%\end{itemize}

\subsection{Special considerations}

The core task in this test is having a robot that autonomously tidies up a messed room as a human would do. 

This rulebook presents a series of guidelines and regulations that objectively allows evaluating the different aspects of the 4S philosophy (Speed, Smooth/Smart, Stable, Safe) and therefore the expected behaviour of a robot is not limited to the rules established here. 

In principle, any robot behaviour not regulated here is allowed as long as it doesn't interfere with solving the core task. However, the Technical Committee is allowed to ban any behaviour not mentioned here and that they may consider unfair (e.g. taking several objects at the same time without announcing it and without a clear object recognition; asking for human interaction beyond the established in the rulebook; etc). If you have any concern about a behaviour that may seem controversial, please contact the TC before considering their application to solving the task.

\iffalse
\newpage
%\pagebreak
\subsection{Score Sheet}

Total score per match:

%\begin{table}
\begin{center}
\begin{tabular}{ |m{8cm}|m{3cm}|m{1cm}| } 
    \hline
    \textbf{Performance} & \textbf{Score} & \textbf{Points} \\ 
    \hline
    \textbf{\textit{Task 1}} & & \\ 
    \hline
    Grasping an object inside the room & 10 x 30 & \\ 
    \hline
    Correctly depositing an object in a location & 10 x 30 & \\ 
    \hline
    Correctly depositing an object in a location according to its category & +10 per object & \\ 
    \hline
    Correctly depositing an object based on its orientation (e.g. cups facing upwards, pencil tips facing downwards) & +10 per object & \\
    \hline
    Finishing the task within the time limit (15 min $\pm$ 15 sec) & 50 points & \\ 
    \hline
    & & \\ 
    \hline
    \textbf{\textit{Task 2a}} & & \\ 
    \hline
    Successfully entering the arena without collisions & 100 & \\ 
    \hline
    Hitting an obstacle & -50 per hit & \\ 
    \hline
    \textbf{\textit{Task 2b}} & & \\ 
    \hline
    Taking the target object among many objects in the shelf & 100 & \\ 
    \hline
    Taking the wrong object & -70 & \\ 
    \hline
    Correctly detecting a person request and giving the drink to her/him & 100 & \\ 
    \hline
    Delivering the object to the wrong person & -70 & \\ 
    \hline
    Finishing the task within time limit (5 min $\pm$ 15 sec) & 50 & \\ 
    \hline
     & & \\ 
    \hline
    \textbf{\textit{Special Bonuses}} & & \\ 
    \hline
    If any time remaining, add 1 points per minute & 20 x time & \\ 
    \hline
    Bonus challenges (Hard-to-grasp object manipulation, Opening the three drawers) & 50 per challenge & \\ 
    \hline
    \textbf{\textit{Penalties}} & & \\ 
    \hline
    Moving to the next room before time & -100 & \\ 
    \hline
    Hitting the furniture or the objects/obstacles inside the arena & -50 per hit & \\ 
    \hline
    Dropping an object without re-taking it & -10 per object & \\ 
    \hline
    Performing a false delivery & -10 per delivery & \\ 
    \hline
    Restart (within one minute) & -100 & \\ 
    \hline
    Additional restarting time & -100 per minute & \\ 
    \hline
    \hline
    \textbf{Total (no penalties/bonuses considered)} & & 1000 \\ 
    \hline
\end{tabular}
\end{center}
%\label{table:scores}
%\end{table}
\fi

\newpage
%\pagebreak
\subsection{Score Sheet - Task 1}

Total score per match:\\ \\

\begin{center} Team name/number: $\rule{6cm}{0.15mm}$ \end{center}

%\begin{table}
\begin{center}
\begin{tabular}{|m{0.4cm}|m{0.8cm}|m{1.2cm}|m{1cm}|m{1cm}|m{0.8cm}|m{0.8cm}|m{1cm}|m{0.8cm}|m{0.8cm}|}
    \hline
    &\multicolumn{4}{|c|}{Penalties }& \multicolumn{5}{|c|}{Points }\\
    \cline{2-10}
    & & &\multicolumn{2}{|c|}{Hit}&\multicolumn{3}{|c|}{Correct}& &\\
    \cline{4-5} \cline{6-8}
    \#& Drop& Restart& Object& Furni- ture& Deli- very& Cate- gory& Orien- tation& False& Total \\
    \hline
     & \multicolumn{1}{|c|}{x0.5} & \multicolumn{1}{|c|}{x0.5} & \multicolumn{1}{|c|}{x0.5} & \multicolumn{1}{|c|}{x0.5}& 10& +10& +10& x0.0&\\
    \cline{2-9}
    1 & \multicolumn{8}{|l|}{Notes: }&\\
    \hline
     & \multicolumn{1}{|c|}{x0.5} & \multicolumn{1}{|c|}{x0.5} & \multicolumn{1}{|c|}{x0.5} & \multicolumn{1}{|c|}{x0.5}& 10& +10& +10& x0.0&\\
    \cline{2-9}
    2 & \multicolumn{8}{|l|}{Notes: }&\\
    \hline
     & \multicolumn{1}{|c|}{x0.5} & \multicolumn{1}{|c|}{x0.5} & \multicolumn{1}{|c|}{x0.5} & \multicolumn{1}{|c|}{x0.5}& 10& +10& +10& x0.0&\\
    \cline{2-9}
    3 & \multicolumn{8}{|l|}{Notes: }&\\
    \hline
     & \multicolumn{1}{|c|}{x0.5} & \multicolumn{1}{|c|}{x0.5} & \multicolumn{1}{|c|}{x0.5} & \multicolumn{1}{|c|}{x0.5}& 10& +10& +10& x0.0&\\
    \cline{2-9}
    4 & \multicolumn{8}{|l|}{Notes: }&\\
    \hline
     & \multicolumn{1}{|c|}{x0.5} & \multicolumn{1}{|c|}{x0.5} & \multicolumn{1}{|c|}{x0.5} & \multicolumn{1}{|c|}{x0.5}& 10& +10& +10& x0.0&\\
    \cline{2-9}
    5 & \multicolumn{8}{|l|}{Notes: }&\\
    \hline
     & \multicolumn{1}{|c|}{x0.5} & \multicolumn{1}{|c|}{x0.5} & \multicolumn{1}{|c|}{x0.5} & \multicolumn{1}{|c|}{x0.5}& 10& +10& +10& x0.0&\\
    \cline{2-9}
    6 & \multicolumn{8}{|l|}{Notes: }&\\
    \hline
     & \multicolumn{1}{|c|}{x0.5} & \multicolumn{1}{|c|}{x0.5} & \multicolumn{1}{|c|}{x0.5} & \multicolumn{1}{|c|}{x0.5}& 10& +10& +10& x0.0&\\
    \cline{2-9}
    7 & \multicolumn{8}{|l|}{Notes: }&\\
    \hline
     & \multicolumn{1}{|c|}{x0.5} & \multicolumn{1}{|c|}{x0.5} & \multicolumn{1}{|c|}{x0.5} & \multicolumn{1}{|c|}{x0.5}& 10& +10& +10& x0.0&\\
    \cline{2-9}
    8 & \multicolumn{8}{|l|}{Notes: }&\\
    \hline
     & \multicolumn{1}{|c|}{x0.5} & \multicolumn{1}{|c|}{x0.5} & \multicolumn{1}{|c|}{x0.5} & \multicolumn{1}{|c|}{x0.5}& 10& +10& +10& x0.0&\\
    \cline{2-9}
    9 & \multicolumn{8}{|l|}{Notes: }&\\
    \hline
     & \multicolumn{1}{|c|}{x0.5} & \multicolumn{1}{|c|}{x0.5} & \multicolumn{1}{|c|}{x0.5} & \multicolumn{1}{|c|}{x0.5}& 10& +10& +10& x0.0&\\
    \cline{2-9}
    10 & \multicolumn{8}{|l|}{Notes: }&\\
    \hline
     & \multicolumn{1}{|c|}{x0.5} & \multicolumn{1}{|c|}{x0.5} & \multicolumn{1}{|c|}{x0.5} & \multicolumn{1}{|c|}{x0.5}& 10& +10& +10& x0.0&\\
    \cline{2-9}
    11 & \multicolumn{8}{|l|}{Notes: }&\\
    \hline
     & \multicolumn{1}{|c|}{x0.5} & \multicolumn{1}{|c|}{x0.5} & \multicolumn{1}{|c|}{x0.5} & \multicolumn{1}{|c|}{x0.5}& 10& +10& +10& x0.0&\\
    \cline{2-9}
    12 & \multicolumn{8}{|l|}{Notes: }&\\
    \hline
     & \multicolumn{1}{|c|}{x0.5} & \multicolumn{1}{|c|}{x0.5} & \multicolumn{1}{|c|}{x0.5} & \multicolumn{1}{|c|}{x0.5}& 10& +10& +10& x0.0&\\
    \cline{2-9}
    13 & \multicolumn{8}{|l|}{Notes: }&\\
    \hline
     & \multicolumn{1}{|c|}{x0.5} & \multicolumn{1}{|c|}{x0.5} & \multicolumn{1}{|c|}{x0.5} & \multicolumn{1}{|c|}{x0.5}& 10& +10& +10& x0.0&\\
    \cline{2-9}
    14 & \multicolumn{8}{|l|}{Notes: }&\\
    \hline
     & \multicolumn{1}{|c|}{x0.5} & \multicolumn{1}{|c|}{x0.5} & \multicolumn{1}{|c|}{x0.5} & \multicolumn{1}{|c|}{x0.5}& 10& +10& +10& x0.0&\\
    \cline{2-9}
    15 & \multicolumn{8}{|l|}{Notes: }&\\
    \hline
    \multicolumn{10}{|l|}{Comments:} \\ 
    \multicolumn{10}{|l|}{} \\
    \multicolumn{10}{|l|}{} \\
    \hline
\end{tabular}
\end{center}
%\label{table:scores}
%\end{table}

\newpage

%\begin{table}
\begin{center}
\begin{tabular}{|m{0.4cm}|m{0.8cm}|m{1cm}|m{1cm}|m{1.2cm}|m{0.8cm}|m{0.8cm}|m{1cm}|m{0.8cm}|m{0.8cm}|}
    \hline
    &\multicolumn{4}{|c|}{Penalties }& \multicolumn{5}{|c|}{Points }\\
    \cline{2-10}
    & & &\multicolumn{2}{|c|}{Hit}&\multicolumn{3}{|c|}{Correct}& &\\
    \cline{4-5} \cline{6-8}
    \#& Drop& Restart& Object& Furni- ture& Deli- very& Cate- gory& Orien- tation& False& Total \\
    \hline
     & \multicolumn{1}{|c|}{x0.5} & \multicolumn{1}{|c|}{x0.5} & \multicolumn{1}{|c|}{x0.5} & \multicolumn{1}{|c|}{x0.5}& 10& +10& +10& x0.0&\\
    \cline{2-9}
    16 & \multicolumn{8}{|l|}{Notes: }&\\
    \hline
     & \multicolumn{1}{|c|}{x0.5} & \multicolumn{1}{|c|}{x0.5} & \multicolumn{1}{|c|}{x0.5} & \multicolumn{1}{|c|}{x0.5}& 10& +10& +10& x0.0&\\
    \cline{2-9}
    17 & \multicolumn{8}{|l|}{Notes: }&\\
    \hline
     & \multicolumn{1}{|c|}{x0.5} & \multicolumn{1}{|c|}{x0.5} & \multicolumn{1}{|c|}{x0.5} & \multicolumn{1}{|c|}{x0.5}& 10& +10& +10& x0.0&\\
    \cline{2-9}
    18 & \multicolumn{8}{|l|}{Notes: }&\\
    \hline
     & \multicolumn{1}{|c|}{x0.5} & \multicolumn{1}{|c|}{x0.5} & \multicolumn{1}{|c|}{x0.5} & \multicolumn{1}{|c|}{x0.5}& 10& +10& +10& x0.0&\\
    \cline{2-9}
    19 & \multicolumn{8}{|l|}{Notes: }&\\
    \hline
     & \multicolumn{1}{|c|}{x0.5} & \multicolumn{1}{|c|}{x0.5} & \multicolumn{1}{|c|}{x0.5} & \multicolumn{1}{|c|}{x0.5}& 10& +10& +10& x0.0&\\
    \cline{2-9}
    20 & \multicolumn{8}{|l|}{Notes: }&\\
    \hline
     & \multicolumn{1}{|c|}{x0.5} & \multicolumn{1}{|c|}{x0.5} & \multicolumn{1}{|c|}{x0.5} & \multicolumn{1}{|c|}{x0.5}& 10& +10& +10& x0.0&\\
    \cline{2-9}
    21 & \multicolumn{8}{|l|}{Notes: }&\\
    \hline
     & \multicolumn{1}{|c|}{x0.5} & \multicolumn{1}{|c|}{x0.5} & \multicolumn{1}{|c|}{x0.5} & \multicolumn{1}{|c|}{x0.5}& 10& +10& +10& x0.0&\\
    \cline{2-9}
    22 & \multicolumn{8}{|l|}{Notes: }&\\
    \hline
     & \multicolumn{1}{|c|}{x0.5} & \multicolumn{1}{|c|}{x0.5} & \multicolumn{1}{|c|}{x0.5} & \multicolumn{1}{|c|}{x0.5}& 10& +10& +10& x0.0&\\
    \cline{2-9}
    23 & \multicolumn{8}{|l|}{Notes: }&\\
    \hline
     & \multicolumn{1}{|c|}{x0.5} & \multicolumn{1}{|c|}{x0.5} & \multicolumn{1}{|c|}{x0.5} & \multicolumn{1}{|c|}{x0.5}& 10& +10& +10& x0.0&\\
    \cline{2-9}
    24 & \multicolumn{8}{|l|}{Notes: }&\\
    \hline
     & \multicolumn{1}{|c|}{x0.5} & \multicolumn{1}{|c|}{x0.5} & \multicolumn{1}{|c|}{x0.5} & \multicolumn{1}{|c|}{x0.5}& 10& +10& +10& x0.0&\\
    \cline{2-9}
    25 & \multicolumn{8}{|l|}{Notes: }&\\
    \hline
     & \multicolumn{1}{|c|}{x0.5} & \multicolumn{1}{|c|}{x0.5} & \multicolumn{1}{|c|}{x0.5} & \multicolumn{1}{|c|}{x0.5}& 10& +10& +10& x0.0&\\
    \cline{2-9}
    26 & \multicolumn{8}{|l|}{Notes: }&\\
    \hline
     & \multicolumn{1}{|c|}{x0.5} & \multicolumn{1}{|c|}{x0.5} & \multicolumn{1}{|c|}{x0.5} & \multicolumn{1}{|c|}{x0.5}& 10& +10& +10& x0.0&\\
    \cline{2-9}
    27 & \multicolumn{8}{|l|}{Notes: }&\\
    \hline
     & \multicolumn{1}{|c|}{x0.5} & \multicolumn{1}{|c|}{x0.5} & \multicolumn{1}{|c|}{x0.5} & \multicolumn{1}{|c|}{x0.5}& 10& +10& +10& x0.0&\\
    \cline{2-9}
    28 & \multicolumn{8}{|l|}{Notes: }&\\
    \hline
     & \multicolumn{1}{|c|}{x0.5} & \multicolumn{1}{|c|}{x0.5} & \multicolumn{1}{|c|}{x0.5} & \multicolumn{1}{|c|}{x0.5}& 10& +10& +10& x0.0&\\
    \cline{2-9}
    29 & \multicolumn{8}{|l|}{Notes: }&\\
    \hline
     & \multicolumn{1}{|c|}{x0.5} & \multicolumn{1}{|c|}{x0.5} & \multicolumn{1}{|c|}{x0.5} & \multicolumn{1}{|c|}{x0.5}& 10& +10& +10& x0.0&\\
    \cline{2-9}
    30 & \multicolumn{8}{|l|}{Notes: }&\\
    \hline
    \multicolumn{9}{|l|}{Finishing the task within the time limit (15 min $\pm$ 15 sec): 50 points} & \\
    \hline
    \multicolumn{9}{|l|}{Bonus challenges (50 points per challenge):} &\\
    \multicolumn{9}{|l|}{--Boss-character manipulation} &\\
    \multicolumn{9}{|l|}{--Opening the three drawers (without further human interaction)} &\\
    \hline
    \multicolumn{9}{|r|}{\textbf{SUBTOTAL (Task 1)}  } &\\ 
    \hline
    \multicolumn{10}{|l|}{Comments:} \\ 
    \multicolumn{10}{|l|}{} \\
    \multicolumn{10}{|l|}{} \\
    \hline
\end{tabular}
\end{center}
%\label{table:scores}
%\end{table}

\begin{center} Referee: $\rule{3cm}{0.15mm}$ Team Leader: $\rule{3cm}{0.15mm}$ \end{center}

\newpage
%\pagebreak

\subsection{Score Sheet - Task 2}

Total score per match:\\ \\

\begin{center} Team name/number: $\rule{6cm}{0.15mm}$ \end{center}

%\begin{table}
\begin{center}
\begin{tabular}{ |m{8cm}|m{3cm}|m{1cm}| } 
    \hline
    \multicolumn{3}{|l|}{\textbf{\textit{Task 2a}}} \\ 
    \hline
    Successfully entering to the dining room to the goal location and without collisions & 100 & \\ 
    \hline
    \multicolumn{3}{|l|}{\textbf{\textit{Task 2b}}} \\ 
    \hline
    Taking any food item in the shelf & 40 & \\ 
    \hline
    Taking the requested object among many objects in the shelf & +60 & \\ 
    \hline
    \multicolumn{2}{|r|}{SUM$_1$  } &\\
    \hline
    \multicolumn{3}{|l|}{HIT$_1$:\hspace*{2cm} 1\hspace*{2cm} 2\hspace*{2cm} 3\hspace*{2cm} 4+} \\ 
    \hline
    \multicolumn{2}{|r|}{Task\_2b$_1$ = (1 - 0.25xHIT$_1$)xSUM$_1$} &\\
    \hline
    \multicolumn{3}{|l|}{}\\
    \hline
    Delivering the object to a person in the delivery area & 40 & \\ 
    \hline
    Correctly detecting a person's request and giving the item to her/him & +60 & \\ 
    \hline
    \multicolumn{2}{|r|}{SUM$_2$  } &\\
    \hline
    \multicolumn{3}{|l|}{HIT$_2$:\hspace*{2cm} 1\hspace*{2cm} 2\hspace*{2cm} 3\hspace*{2cm} 4+} \\ 
    \hline
    \multicolumn{2}{|r|}{Task\_2b$_2$ = (1 - 0.25xHIT$_2$)xSUM$_2$} &\\
    \hline
    \multicolumn{3}{|l|}{}\\
    \hline
    \multicolumn{2}{|r|}{SUM = Task\_2a + Task\_2b$_1$ + Task\_2b$_2$  } &\\
    \hline
    \multicolumn{3}{|l|}{}\\
    \hline
    Finishing the task within the time limit (5 min $\pm$ 15 sec) & 50 & \\ 
    \hline
    If any time remaining, add 20 points per minute & 20 x minutes & \\ 
    \hline
    \multicolumn{2}{|r|}{\textbf{SUBTOTAL (Task 2)}  } &\\
    \hline
    \multicolumn{3}{|l|}{}\\
    \hline
    \multicolumn{2}{|r|}{\textbf{TOTAL (Task 1 + Task 2)}  } &\\
    \hline
    \multicolumn{3}{|l|}{}\\
    \hline
    \multicolumn{3}{|l|}{\textbf{\textit{Special Penalties}}} \\ 
    \hline
    Three or more restarts & Total points x 0.0 & \\
    \hline
    Emergency stop by a referee & Total points x 0.0 & \\
    \hline
    \multicolumn{3}{|l|}{}\\
    \hline
    \multicolumn{3}{|l|}{Note: a HIT considers dropping an object out of the shelf, a restart, and a regular collision.} \\ 
    \hline
\end{tabular}
\end{center}
%\label{table:scores}
%\end{table}

\begin{center} Referee: $\rule{3cm}{0.15mm}$ Team Leader: $\rule{3cm}{0.15mm}$ \end{center}

\newpage

Total points per match (win: 3, draw: 1, defeat: 0):

\begin{table}[h!]
\begin{center}
\begin{tabular}{ |m{1.5cm}|m{0.6cm}|m{0.6cm}|m{0.6cm}|m{0.6cm}|m{0.6cm}|m{0.6cm}|m{0.6cm}|m{0.6cm}| } 
    \hline
    &\multicolumn{8}{|c|}{\textbf{Team}}\\
    \cline{2-9}
    &\multicolumn{4}{|c|}{\textbf{Group A}} & \multicolumn{4}{|c|}{\textbf{Group B}}\\
	\cline{2-9}
    \textbf{Match} & \textbf{I} & \textbf{II} & \textbf{III} & \textbf{IV} & \textbf{V} & \textbf{VI} & \textbf{VII} & \textbf{VII} \\ 
    \hline
    \textbf{1} & & & & & & & & \\
    \hline
    \textbf{2} & & & & & & & & \\
    \hline
    \textbf{3} & & & & & & & & \\
    \hline
    \textbf{Total} & & & & & & & & \\
    \hline
    \hline
    & \multicolumn{2}{|c|}{\textbf{S1}} & \multicolumn{2}{|c|}{\textbf{S2}} & \multicolumn{2}{|c|}{\textbf{S3}} & \multicolumn{2}{|c|}{\textbf{S4}} \\
    \hline
    \textbf{1} & \multicolumn{2}{|c|}{} & \multicolumn{2}{|c|}{} & \multicolumn{2}{|c|}{} & \multicolumn{2}{|c|}{} \\
    \hline
    \hline
    & \multicolumn{4}{|c|}{\textbf{Final}} & \multicolumn{4}{|c|}{\textbf{3rd place}} \\
    \hline
    & \multicolumn{2}{|c|}{\textbf{F1}} & \multicolumn{2}{|c|}{\textbf{F2}} & \multicolumn{2}{|c|}{\textbf{F4}} & \multicolumn{2}{|c|}{\textbf{F4}} \\
    \hline
    \textbf{1} & \multicolumn{2}{|c|}{} & \multicolumn{2}{|c|}{} & \multicolumn{2}{|c|}{} & \multicolumn{2}{|c|}{} \\
    \hline
\end{tabular}
\end{center}
\label{table:fixtures}
\end{table}

NOTE: If there are more than eight participating teams, a selection tournament might take place during the competition. We will announce the rules and considerations in due time. 

\newpage
\section{Appendix}
\subsection{Furniture models} \label{sec:furniture}

Note: units are in meters in the following order:  width x length x height.

\begin{itemize}
    \item stairs-like drawer (white structure, orange drawers):
    
    Link: \url{https://www.ikea.com/us/en/catalog/products/S89857541/#/S59128935}

    \item long table (white):

    Link: \url{https://www.ikea.com/us/en/catalog/products/00342640/}

    \item tall table (white):

    Link: \url{https://www.ikea.com/us/en/catalog/products/20342644/#/60342642}

    \item bin (two bins, one green and the other black):

    Link: \url{https://www.ikea.com/us/en/catalog/products/10300319/}

    \item  shelf (white):

    Link: \url{https://www.ikea.com/us/en/catalog/products/00263850/}

    \item chair (white structure, gray cover):

    Link: \url{https://www.ikea.com/us/en/catalog/products/S69100163/}
\end{itemize}

\begin{table}[h!]
\begin{center}
\begin{tabular}{ |M{2.0cm}|M{2.0cm}|M{1.0cm}|M{1.0cm}|M{1.0cm}|M{1.5cm}| }
    \hline
    Name & IKEA's Model & width & length & height & Image\\
    \hline
    stairs-like drawer & 591.289.35 & 0.44m & 0.99m & 0.94m & Figure \ref{fig:drawermodel}\\
    \hline
    long table & 003.426.40 & 0.40m & 1.20m & 0.40m & Figure \ref{fig:longtablemodel}\\
    \hline
    tall table & 603.426.42 & 0.40m & 0.40m & 0.60m & Figure \ref{fig:talltablemodel}\\
    \hline
    bin & 103.003.19 & 0.38m & 0.33m & 0.33m & Figure \ref{fig:binmodel}\\
    \hline
    shelf & 002.638.50 & 0.28m & 0.80m & 2.02m & Figure \ref{fig:shelfmodel}\\
    \hline
    chair & 691.001.63 & 0.58m & 0.54m & 0.97m & Figure \ref{fig:chairmodel}\\
    \hline
\end{tabular}
\end{center}
\caption{Furniture's model and size to be used during the tests and competition.}
\label{table:furniture}
\end{table}

\newpage

\begin{figure}[!h]
  	\centering
  	\includegraphics[width=0.80\textwidth]{{images/model_drawer.png}}
	\caption{Stairs-like drawer model (width 0.44 m, length 0.99 m, height 0.94 m).}
	\label{fig:drawermodel}
\end{figure}

\begin{figure}[!h]
  	\centering
  	\includegraphics[width=0.80\textwidth]{{images/model_longtable.png}}
	\caption{Long table model (width 0.40, length 1.20, height 0.40 m).}
	\label{fig:longtablemodel}
\end{figure}

\begin{figure}[!h]
  	\centering
  	\includegraphics[width=0.80\textwidth]{{images/model_talltable.png}}
	\caption{Tall table model (width 0.40, length 0.40, height 0.60).}
	\label{fig:talltablemodel}
\end{figure}

\begin{figure}[!h]
	\centering
  	\includegraphics[width=0.80\textwidth]{{images/model_bin.png}}
	\caption{Bin model (width 0.38, length 0.33, height 0.33).}
	\label{fig:binmodel}
\end{figure}

\begin{figure}[!h]
	\centering
  	\includegraphics[width=0.80\textwidth]{{images/model_shelf.png}}
	\caption{Shelf model (width 0.28, length 0.80, height 2.02).}
	\label{fig:shelfmodel}
\end{figure}

\begin{figure}[!h]
	\centering
  	\includegraphics[width=0.80\textwidth]{{images/model_chair.png}}
	\caption{Chair model (width 0.58, length 0.54, height 0.97).}
	\label{fig:chairmodel}
\end{figure}

\newpage
\subsection{Arena Setup}

The arena setup consists of two rooms (Room\_1 and Room\_2), one per task, as in Figure \ref{fig:arenaareas}. 

\subsubsection{Room 1}

In Task\_1, the room is divided in two main areas, the \textbf{Search Area}, where the objects are scattered around, and the \textbf{Deposit Area}, where the objects should be placed.

\begin{figure}[!h]%[!htbp]
	\centering
	%\begin{center}
	\includegraphics[width=0.80\textwidth]{{images/map_areas.png}}
	\caption{Test arena by interaction areas. There are two rooms, one per task, and five main areas: for Task 1, in Room\_1, we have the Deposit and Search areas; in the case of Task 2 (Room\_2), for Task 2a we have the Obstacle Avoidance and Goal areas and, for Task 2b, the Food and Delivery areas.}
	\label{fig:arenaareas}
	%\end{center}
\end{figure}

The furniture regarding the Deposit Area includes the Drawer, Long Table, Bin\_A, and Bin\_B. In detail, the Drawer is divided in Drawer\_left, Drawer\_top, and Drawer\_bottom (see Figure \ref{fig:drawerstags}).

\begin{figure}[!h]%[!htbp]
	\centering
	%\begin{center}
	\includegraphics[width=0.80\textwidth]{{images/drawers_tags.png}}
	\caption{In the stairs-like drawer, only three drawers are going to be used, namely the Drawer\_left, and middle Drawer\_top and Drawer\_bottom.}
	\label{fig:drawerstags}
	%\end{center}
\end{figure}

Furthermore, the Long Table includes two types of deposits: tray and container. There are two similar trays (Tray\_A and Tray\_B) and two containers (Container\_A and Container\_B). The Container\_B should be small enough to contain markers/cutlery vertically in order to be able to evaluate the correct orientation and therefore we use the IKEA's Gessan container, as in \url{https://www.ikea.com/us/en/p/gessan-box-white-10371825/}.

\begin{figure}[!h]%[!htbp]
	\centering
	%\begin{center}
	\includegraphics[width=0.80\textwidth]{{images/tray_container_tags.jpg}}
	\caption{In the Deposit Area, on the Long Table A there will be four deposits: Tray A, Tray B, Container A, and Container B. As a reference, the tray's size is 0.38 x 0.36 m, Container A is 0.28 x 0.28 x 0.125 m, and Container B is 0.095 x 0.095 m and 0.105 m height. Single color deposits are considered to detect any tag or mark that might be used as a reference.}
	\label{fig:containertags}
	%\end{center}
\end{figure}

A summary of the deposits per category can be found in Table \ref{table:deposits}. As a remainder, the object categories are as follows: Food (e.g. chips can, coffee can, cracker box), kitchen items (e.g. glass, bowl, mug), tools (e.g. screwdrivers, clamps, padlock and keys), shape items (e.g. baseball, tennis ball, foam brick), task items (e.g. Rubick’s cube, t-shirt, airplane toy), orientation-based items (e.g. markers, cutlery), and unknown objects.

\begin{table}[h!]
\begin{center}
\begin{tabular}{ |m{2.5cm}|m{2.5cm}|m{4.0cm}| } 
    \hline
    \begin{center} \textbf{Deposit} \end{center} & 
    \begin{center} \textbf{Place} \end{center} &
    \begin{center} \textbf{Category} \end{center}\\
    \hline
    Drawer\_left & Drawer & Shape items \\
    \hline
    Drawer\_top & Drawer & Tools \\
    \hline
    Drawer\_bottom & Drawer & Tools \\
    \hline
    Tray\_A & Long\_Table\_A & Food \\
    \hline
    Tray\_B & Long\_Table\_A & Food \\
    \hline
    Container\_A & Long\_Table\_A & Kitchen items \\
    \hline
    Container\_B & Long\_Table\_A & Orientation-based items \\
    \hline
    Bin\_A & Bin\_A & Task items\\
    \hline
    Bin\_B & Bin\_B & Unknown objects \\
    \hline
\end{tabular}
\end{center}
\caption{Object categories and their corresponding deposits in the arena, as in Figure \ref{fig:arena}.}
\label{table:deposits}
\end{table}

\subsubsection{Room 2}

Regarding Task 2a, the obstacles will be scattered around the access between rooms (i.e. the Obstacle Avoidance Area in Figure \ref{fig:arenaareas}), and they consist in flat and small objects bellow the laser height (as shown in Figure \ref{fig:obstacles}), therefore, other detection strategies should be implemented to be able to avoid them. The robot should navigate through the Obstacle Avoidance Area to a pre-announced goal location in the Goal Area.

\begin{figure}[!h]%[!htbp]
	\centering
	%\begin{center}
	\includegraphics[width=0.80\textwidth]{{images/obstacles.jpg}}
	\caption{Obstacles laying on the rooms' access. They consist in small and or flat objects.}
	\label{fig:obstacles}
	%\end{center}
\end{figure}

In Task 2b, the food items in the Food Area will be placed inside a shelf (Figure \ref{fig:shelf} and Figure \ref{fig:shelfmodel}) with board levels at 0.50 m, 0.80 m, and 1.05 m height and three different depths: front, middle, and back, as shown in Figure \ref{fig:shelfobjects}. The target object may be at any depth and therefore a grasping strategy should be developed to softly move the objects obstructing the requested drink as necessary (e.g. if the requested item is in the middle level and middle depth, first move any object in the front depth to a different level before being able to grasp the requested food).

\begin{figure}[!h]%[!htbp]
	\centering
	\begin{subfigure}{.45\textwidth}
  		\centering
	    \includegraphics[width=1\textwidth]{{images/shelf_objects_a.jpg}}
	    \caption{}
  		\label{fig:shelfobjectsa}
	\end{subfigure}
	\begin{subfigure}{.45\textwidth}
  		\centering
  		\includegraphics[width=1\textwidth]{{images/shelf_objects_b.jpg}}
  		\caption{}
  		\label{fig:shelfobjectsb}
	\end{subfigure}
	\caption{a) Lateral and b) front views of the objects' distribution at three different depths (front, middle and back).}
	\label{fig:shelfobjects}
	%\end{center}
\end{figure}

Finally, there will be two persons in the Delivery Area (they might be either standing up or sitting down) and one of them will be requesting an item by waving her/his arm (it can be both the left or right arm).

\newpage
\subsection{Objects to be used}

Object's specifications and their corresponding 3D models can be found in \url{http://www.ycbbenchmarks.com} . Mass and weight can be consulted in \url{http://www.ycbbenchmarks.com/wp-content/uploads/2015/09/object-list-Sheet1.pdf}.

IMPORTANT: \textbf{Object availability}: Please consider that some objects might be no longer available and might be replaced by an equivalent object based on availability and similarly in dimensions and visual appearance. For details and updates, please check the following site: \url{https://www.uml.edu/Research/NERVE/YCB-Form.aspx} . We will let you know in advance the dataset to be used during the competition.

\begin{itemize}
    \item Food items:
        \SubItem{Cheez-it cracker box}
        \SubItem{Domino sugar box}
        \SubItem{Jell-o chocolate pudding box}
        \SubItem{Jell-o strawberry gelatin box}
        \SubItem{Spam potted meat can}
        \SubItem{Master chef coffee can}
        \SubItem{Starkist tuna fish can}
        \SubItem{Pringles chips can}
        \SubItem{French's mustard bottle}
        \SubItem{Tomato soup can}
        \SubItem{Plastic banana}
        \SubItem{Plastic strawberry}
        \SubItem{Plastic apple}
        \SubItem{Plastic lemon}
        \SubItem{Plastic peach}
        \SubItem{Plastic pear}
        \SubItem{Plastic orange}
        \SubItem{Plastic plum}
    \item Kitchen items:
        \SubItem{Windex Spray bottle}
        \SubItem{Srub cleanser bottle}
        \SubItem{Scotch brite dobie sponge}
        \SubItem{Pitcher base}
        \SubItem{Pitcher lid}
        \SubItem{Plate}
        \SubItem{Bowl}
        \SubItem{Fork}
        \SubItem{Spoon}
        \SubItem{Spatula}
        \SubItem{Wine glass}
        \SubItem{Mug}
    \item Tool items:
        \SubItem{Large marker}
        \SubItem{Small marker}
        \SubItem{Keys (from the Padlock)}
        \SubItem{Bolt and nut}
        \SubItem{Clamps}
    \item Shape items:
        \SubItem{Credit card blank}
        \SubItem{Mini soccer ball}
        \SubItem{Soft ball}
        \SubItem{Baseball}
        \SubItem{Tennis ball}
        \SubItem{Racquetball}
        \SubItem{Golf ball}
        \SubItem{Marbles}
        \SubItem{Cups}
        \SubItem{Foam brick}
        \SubItem{Dice}
        \SubItem{Rope}
        \SubItem{Chain}
    \item Task items:
        \SubItem{Rubik's cube}
        \SubItem{Colored wood blocks}
        \SubItem{9-peg-hole test}
        \SubItem{Toy airplane}
        \SubItem{Lego duplo}
        \SubItem{Magazine}
        \SubItem{Black t-shirt}
        \SubItem{Timer}






    %\item Hard-to-manipulate:







    \item Discarded:
        \SubItem{Skillet}
        \SubItem{Skillet lid}
        \SubItem{Table cloth}
        \SubItem{Hammer}
        \SubItem{Adjustable wrench}
        \SubItem{Wood block}
        \SubItem{Power drill}
        \SubItem{Washers}
        \SubItem{Nails}
        \SubItem{Knife}
        \SubItem{Scissors}
        \SubItem{Padlock}
        \SubItem{Phillips screwdriver}
        \SubItem{Flat screwdriver}
        \SubItem{Clear box}
        \SubItem{Box lid}
        \SubItem{Footlocker}
\end{itemize}

\end{document}
